\section{小跳蛙问题}

\subsection{问题分析}

\subsubsection{问题设计}

设小跳蛙跳到第$i$块石头上,最小消耗的体力为$p[i]$。

\subsubsection{递推关系建立}
跳蛙可以从第$i-k$到第$i-1$块石头跳到第$i$块石头,设其从第$j$块石头跳,则相应的体力消耗增加$|h_i - h_j|$
可知,有如下递推关系 
\begin{equation}
    \displaystyle
    p[i] = \min_{j = i-k}^{i-1}(p[j] + |h_i - h_j|)
\end{equation}

\subsubsection{自底向上计算(初始化)}

由于小跳蛙开始就在第1块石头上,因此有$p[1] = 0$。之后按照递推式,依次从2到n求出$p[i]$即可。

\subsubsection{目标状态}

由问题定义可知,目标状态即为p[n]。

\subsection{算法描述}
如算法\ref{algo:jump}。

\begin{algorithm}[H]
    \SetKw{Let}{Let}
    \SetKw{Var}{Var}
    \caption{$jump(h[1..n],k)$}\label{algo:jump}
    \KwIn{每块石头高度$h[1..n]$,跳蛙最远一次跳过石头数$k$}
    \KwOut{跳到第n块石头花费的最小体力$p[n]$}
    $p[1] \leftarrow 0$\\
    \For(){ $i \leftarrow 2$ to $n$ }{
       $ p[i] \leftarrow \infty$\\
       \For(){$j\leftarrow \min(i-k,1)$ to $i-1$}{
            $p[i] \leftarrow \min(p[i],p[j] + |h_i-h_j|)$\\
       } 
    }
    \Return{$p[n]$}
\end{algorithm}


\subsection{时间复杂度分析}

从伪代码可知,算法有两层循环,第一层循环复杂度为$O(n)$,第二层循环复杂度为$O(k)$, 因此总算法复杂度为$O(n) * O(k) = O(nk)$,可以发现,若当$k>n$时,第二层循环次数仍不大于$n$,因此我们也可以认为该问题的总体时间复杂度为$O(n^2)$。

\section{二进制串变换问题}

\subsection{问题分析}
\subsubsection{问题设计}
由问题描述可知,只有当相邻两位中,$a$和$b$的值都恰好相反,且这两位的值不相等,使用交换操作的代价才会小于直接取反这两位。

设将$b[1..i]$变为$a[1..i]$的最小代价为$c[i]$,
\subsubsection{递推关系建立}
当$a[i]$与$b[i]$相等,则此位无需操作,当$a[i]$与$b[i]$不等,可行的操作有将$a[i]$取反,若当$a[i]$与$b[i]$不等,且$a[i-1]$与$b[i-1]$不等,且$a[i]$与$a[i-1]$不等,可行的操作有交换$a[i]$与$a[i-1]$。
有递推关系
\begin{equation}
    c[i]= \min
    \left \{
    \begin{aligned}
        &c[i-1] \quad a[i]\text{与}b[i]\text{相等} \\
        &c[i-1]+1 \quad a[i]\text{与}b[i]\text{不相等} \\
        &c[i-2]+1 \quad i>1, a[i]\text{与}b[i]\text{不等},a[i-1]\text{与}b[i-1]\text{不等},a[i]\text{与}a[i-1]\text{不等}\\
    \end{aligned}
    \right .
\end{equation}

\subsubsection{自底向上计算(初始化)}

可知,$i = 0$时,$a[i]$和$b[i]$中没有元素,则$c[0] = 0$。
之后按照从$1$到$n$的顺序依次计算$c[i]$即可。

\subsubsection{目标状态}

由问题定义可知,目标状态即为$c[n]$。

\subsection{算法描述}

如算法\ref{algo:binaryChange}。

\begin{algorithm}[H]
    \SetKw{Let}{Let}
    \SetKw{Var}{Var}
    \caption{$binaryChange(a[1..n], b[1..n])$}\label{algo:binaryChange}
    \KwIn{$a[1..n], b[1..n]$}
    \KwOut{将串a变成串b的最小操作代价}
    $c[0] \leftarrow 0$\\

    \For(){$i \leftarrow 1$to$n$}{
        $c[i] \leftarrow \infty$\\
        \If(){$a[i]=b[i]$}{
            $c[i] \leftarrow \min(c[i],c[i-1])$\\
        }
        \Else{
            \If(){ $i > 1 \quad and \quad a[i-1] \neq b[i-1]\quad and \quad a[i] \neq a[i-1] $}{
                $c[i]\leftarrow \min(c[i], c[i-2]+1)$\\
            }
            $c[i] \leftarrow min(c[i], c[i-1] + 1 )$\\
        }
    }
    \Return{$c[n]$}
    
\end{algorithm}

\subsection{时间复杂度分析}
该算法共有$n$种状态,状态之间的转移复杂度为$O(1)$,因此总复杂度为$O(n)$。

\section{球队组建问题}

\subsection{问题分析}

\subsubsection{问题设计}

设$C[i,j]$为在前$j$列中选择第$i$行$j$列球员的情况下的身高最大值。

\subsubsection{递推关系建立}

由问题描述可知,前$j$列中选择第$i$行$j$列球员,只会影响第$j$列和第$j-1$列的选择。
由问题设计可知,选择了第$i$行第$j$列球员,就无法选择第$3-i$行第$j$列球员,和第$i$行第$j-1$列球员,
而可以选择第$3-i$行第$j-1$列球员,或不选择第$j-1$列球员,而选择第$3-i$行第$j-2$列球员
(选择第$i$行$j-2$列球员属于选择$3-i$行第$j-1$列球员的子问题)。

因此有

\begin{equation}
    C[i,j] = \max
    \left \{
        \begin{aligned}
            C[3-i,j-1] + h[i,j]\\
            C[3-i,j-2] + h[i,j]
        \end{aligned}
     \right .   
\end{equation}

\subsubsection{自底向上计算(初始化)}

由问题设计易知,$ C[1,0] = 0, C[2,0] = 0, C[1,1] = h[1,1], C[2, 1] = h[2,1] $,
随后按照列从小到大的顺序计算即可。

\subsubsection{目标状态}
由题意可知,最终选择的队员的倒数第二列可能有两种状态,一种是都未选择,此时选择最后一列最高球员为最优解,一种是选择了一位球员,此时选择最后一列
与之不同行的球员为最优解,因此最优解中必定选择了最后一列的球员,因此只需要比较 $C[1,n]$与$C[2,n]$,两者中的最大值即为所求。

\subsection{算法描述}

如算法\ref{algo:choose}

\begin{algorithm}[H]
    \SetKw{Let}{Let}
    \SetKw{Var}{Var}
    \caption{$choose(h[1..2,1..n])$}\label{algo:choose}
    \KwIn{同学身高$h[1..2,1..n]$}
    \KwOut{身高之和最大的队伍的身高$C$}
    
    $C[1,0] \leftarrow 0$\\
    $C[2,0] \leftarrow 0$\\
    $C[1,1] \leftarrow h[1,1]$\\
    $C[2,1] \leftarrow h[2,1]$\\

    \For(){$j \leftarrow 1$ to $n$ }{
        \For(){$i \leftarrow 1$ to $2$}{
            $C[i,j] \leftarrow \max(C[3-i,j-1] + h[i,j] , C[3-i,j-2] + h[i,j])$\\
        }
    }
    \Return{$\max(C[1, n], C[2,n])$}
\end{algorithm}

\subsection{时间复杂度分析}

算法一共有$2n$种状态,状态之间的转移时间为$O(1)$,因此总时间复杂度为$O(n)$。

\section{括号匹配问题}

\subsection{问题分析}

\subsubsection{问题设计}
设$c[i..j]$是第$i$个字符到第$j$个字符中最长合法子序列的长度。

\subsubsection{递推关系建立}

设括号串为$s[1..n]$,
由题目描述可知,合法的序列有两种,一种是序列的两端的括号匹配,另一种是由匹配的序列拼接而成。
若$s[i]$与$s[j]$ 匹配,则一种可能的$c[i..j] = c[i+1..j-1] + 2$。

则有递推关系

\begin{equation}
    c[i,j]= \max
    \left \{
    \begin{aligned}
        &c[i+1, j-1] + 2 \quad s[i] \text{与} s[j]\text{匹配}\\
        &c[i,k]+c[k+1,j]\quad  i<k<j\\
    \end{aligned}
    \right .
\end{equation}

\subsubsection{自底向上计算(初始化)}
易知,区间长度为1的序列不可能存在匹配的括号子序列,因此所有区间为1的$c[i,i] = 0$。

由递推式可知,$c[i,j]$与所有长度小于其的区间最长子序列长度都有关,因此需要按照区间长度由短到长的顺序计算。

\subsubsection{目标状态}
由题目描述可知,即为$c[1,n]$。

\subsection{算法描述}

如算法\ref{algo:match}

\begin{algorithm}[H]
    \SetKw{Let}{Let}
    \SetKw{Var}{Var}
    \caption{$match(s[1..n])$}\label{algo:match}
    \KwIn{括号串$s[1..n]$}
    \KwOut{最长合法子序列}
    
    \For(){$i \leftarrow 1$ to $n$}{
        $c[i,i] \leftarrow 0$\\
    }

    \For(){$l \leftarrow 2$ to $n$}{
        \For(){$i \leftarrow 1$ to $(n-l+1)$}{
            $c[i,i+l-1]\leftarrow 0$\\
            \If(){$ s[i] \text{与} s[i+l-1]\text{匹配}$}{
                $c[i,i+l-1]\leftarrow \max(c[i,i+l-1], c[i+1,i+l-2])$\\
            }
            \For(){$k \leftarrow  i+1 $ to $(i+l-2)$}{
                $c[i,i+l-1]\leftarrow \max(c[i,i+l-1], c[i,k]+c[k+1, i+l-1])$
            }
        }
    }
    \Return{$c[1,n]$}
\end{algorithm}

\subsection{时间复杂度分析}

由算法可知,该算法的状态数有$O(n^2)$种,状态之间的转移复杂度为$O(n)$,因此总时间复杂度为$O(n^3)$。

\section{箱子问题}
\subsection{问题分析}
\subsubsection{问题设计}
由于箱子可以重复使用,并且必须长宽严格小于下面箱子的箱子才能被使用,因此一个箱子最多被使用3次。因此可以将一个箱子看成3个底面不同的箱子(设$w < d$),此时箱子被规定了摆放方式。
问题转化成有 $3n$ 个箱子,摆放方式固定,求最大摆放方式。
将箱子先按照底面面积从大到小排序,再重新编号成$1..3n$。

设$MH[i]$是编号为$i$的箱子作为顶端箱子时,箱子堆的最大高度。

\subsubsection{递推关系建立}

因为只有底面比箱子$i$更大,并且边长满足条件的箱子才能排在第$i$个箱子的下面,有如下递推式

\begin{equation}
    MH[i] =\max \left \{
        \begin{aligned}
            &\max (MH[j] + h[i]) \quad j < i \text{且} box[j].w > box[i].w \text{且} box[j].d > box[i].d \\
            &h[i]\\
        \end{aligned}
    \right .
\end{equation}

\subsubsection{自底向上计算(初始化)}

由问题描述,可知,没有比第一个箱子底面更大的箱子,因此 $MH[1] = box[1].h$,随后按照$i$从小到大的顺序计算即可。

\subsubsection{目标状态}

遍历$MH[1..3n]$,找到$\max(MH[1..3n])$,即为所求。

\subsubsection{路径记录}
用$Under[i]$表示$MH[i]$取最大时,底下的箱子编号,若底下没有箱子,则为0。设$MH$在$i$处取到最大值,则箱子的顺序从上到下为
$i, Under[i], Under[Under[i]]...$,直到最下层箱子 $Under[k] = 0$。最后逆序输出,即可得到题目中要求的从最底层到最顶层的箱子规格。

\subsection{算法描述}

如算法\ref{algo:box}。

\begin{algorithm}[htbp]
    \SetKw{Let}{Let}
    \SetKw{Var}{Var}
    \caption{$maxBoxHight(a[1..n])$}\label{algo:box}
    \KwIn{n个箱子的规格数据$a[1..n]$}
    \KwOut{最高的堆箱子高度$MaxHight$和堆箱子方法$Method$}
    \For(){$i \leftarrow 1$ to $n$}{ \emph{将$n$个箱子转化成$3n$个箱子} \\
        $box[3i-2].h \leftarrow a[i].h$\\
        $box[3i-2].w, box[3i-2].d \leftarrow \min(a[i].w,a[i].d), \max(a[i].w, a[i].d)$\\
        $box[3i-1].h \leftarrow a[i].w$\\
        $box[3i-1].w, box[3i-1].d \leftarrow \min(a[i].h,a[i].d), \max(a[i].h, a[i].d)$\\
        $box[3i].h \leftarrow a[i].d$\\
        $box[3i].w, box[3i-1].d \leftarrow \min(a[i].h,a[i].w), \max(a[i].h, a[i].w)$\\
    }
    $sort(box[1..3n])$ \emph{将$3n$个箱子按照底面积排序} \\
    
    $MH[1] \leftarrow box[1].h, Under[1] \leftarrow 0$ \emph{初始化} \\
    \For(){$i \leftarrow 2$ to $3n$}{ \emph{自底向上计算} \\
        $MH[i] \leftarrow box[i].h$\\
        $Under[i] \leftarrow 0$\\
        \For(){$j \leftarrow 1$ to $i - 1$}{
            \If(){$box[j].w > box[i].w$ and $box[j].d > box[i].d$}{
                \If(){$MH[j] + box[i].h > MH[i]$}{
                    $MH[i] \leftarrow MH[j] + box[i].h$\\
                    $Under[i] \leftarrow j$\\
                }
            }
        }
    }
    $MaxHight \leftarrow 0, index \leftarrow 0$ \emph{找目标状态} \\
    \For(){$i \leftarrow 1$ to $3n$}{
        \If(){$MH[i] > MaxHight$}{
            $MaxHight \leftarrow MH[i]$\\
            $index \leftarrow i$\\
        }
    }
    $cnt \leftarrow 1$   \emph{方案求解}\\
    $Method[cnt] \leftarrow box[index]$\\
    $cnt \leftarrow cnt + 1$\\
    \While(){$Under[index] != 0$}{
        $index \leftarrow Under[index]$\\
        $Method[cnt] \leftarrow box[index]$\\
        $cnt \leftarrow cnt + 1$\\
    }
    \Return{$MaxHight, Method[1..cnt-1]$}
\end{algorithm}

\subsection{时间复杂度分析}
转化$3n$个箱子复杂度为$O(n)$,排序箱子复杂度为$O(n \log n)$,状态数共 $3n$,状态转移复杂度为$O(n)$, 寻找目标状态复杂度为$O(n)$,求解输出方案复杂度为$O(n)$,
因此总时间复杂度为$O(n) + O(n\log n) + O(n) * O(n) + O(n) + O(n) = O(n^2)$。