\section{小跳蛙问题}

\subsection{问题分析}

\subsubsection{问题设计}

设小跳蛙跳到第$i$块石头上,最小消耗的体力为$p[i]$。

\subsusbsection{递推关系建立}
跳蛙可以从第$i-k$到第$i-1$块石头跳到第$i$块石头,设其从第$j$块石头跳,则相应的体力消耗增加$|h_i - h_j|$
可知,有如下递推关系 
\begin{equation}
    \displaystyle
    p[i] = \min_{j = i-k}^{i-1}(p[j] + |h_i - h_j|)
\end{equation}

\subsubsection{自底向上计算(初始化)}

由于小跳蛙开始就在第1块石头上,因此有$p[1] = 0$。之后按照递推式,依次从2到n求出$p[i]$即可。

\subsubsection{目标状态}

由问题定义可知,目标状态即为p[n]。

\subsection{算法描述}
如算法\ref{algo:jump}。

\begin{algorithm}[H]
    \SetKw{Let}{Let}
    \SetKw{Var}{Var}
    \caption{$jump(h[1..n],k)$}\label{algo:jump}
    \KwIn{每块石头高度$h[1..n]$,跳蛙最远一次跳过石头数$k$}
    \KwOut{跳到第n块石头花费的最小体力$p[n]$}
    $p[1] \leftarrow 0$\\
    \For(){ $i$ in $2..n$ }{
       $ p[i] \leftarrow \infty$\\
       \For($j$ in $\min(i-k,1)..i-1$){
            $p[i] \leftarrow \min(p[i],p[j] + |h_i-h_j|)$\\
       } 
    }
    \Return{$p[n]$}
\end{algorithm}


\subsection{时间复杂度分析}

从伪代码可知,算法有两层循环,第一层循环复杂度为$O(n)$,第二层循环复杂度为$O(k)$, 因此总算法复杂度为$O(n) * O(k) = O(nk)$

\section{二进制串变换问题}

\subsection*{问题分析}
\subsubsection{问题设计}
由问题描述可知,只有当相邻两位中,$a$和$b$的值都恰好相反,且这两位的值不相等,使用交换操作的代价才会小于直接取反这两位。

设将$b[1..i]$变为$a[1..i]$的最小代价为$c[i]$,
\subsubsection{递推关系建立}
当$a[i]$与$b[i]$相等,则此位无需操作,当$a[i]$与$b[i]$不等,可行的操作有将$a[i]$取反,若当$a[i]$与$b[i]$不等,且$a[i-1]$与$b[i-1]$不等,且$a[i]$与$a[i-1]$不等,可行的操作有交换$a[i]$与$a[i-1]$。
有递推关系
\begin{equation}
    c[i]= 
    \left \{
    \begin{align}
        &c[i-1] \quad a[i]\text{与}b[i]\text{相等} \\
        &c[i-1]+1 \quad a[i]\text{与}b[i]\text{不相等} \\
        &c[i-2]+1 \quad i>1, a[i]\text{与}b[i]\text{不等},a[i-1]\text{与}b[i-1]\text{不等},a[i]\text{与}a[i-1]\text{不等}\\
    \end{align}
    \right .
\end{equation}

\subsubsection{自底向上计算(初始化)}

可知,$i = 0$时,$a[i]$和$b[i]$中没有元素,则$c[0] = 0$。
之后按照从$1$到$n$的顺序依次计算$c[i]$即可。

\subsubsection{目标状态}

由问题定义可知,目标状态即为$c[n]$。

\subsection*{算法描述}
如算法\ref{algo:binaryChange}。

\begin{algorithm}[H]
    \SetKw{Let}{Let}
    \SetKw{Var}{Var}
    \caption{$binaryChange(a[1..n], b[1..n])$}\label{algo:binaryChange}
    \KwIn{$a[1..n], b[1..n]$}
    \KwOut{将串a变成串b的最小操作代价}
    $c[0] \leftarrow 0$\\

    \For(){$i$ in $1..n$}{
        $c[i] \leftarrow \infty$\\
        \If(){$a[i]=b[i]$}{
            $c[i] \leftarrow \min(c[i],c[i-1])$\\
        }
        \Else{
            \If(){ $i > 1 \quad and \quad a[i-1] \neq b[i-1]\quad and \quad a[i] \neq a[i-1] $}{
                $c[i]\leftarrow \min(c[i], c[i-2]+1)$\\
            }
            $c[i] \leftarrow min(c[i], c[i-1] + 1 )$\\
        }
    }
    \Return{$c[n]$}
    
\end{algorithm}

\subsection*{时间复杂度分析}
该算法共有$n$种状态,状态之间的转移复杂度为$O(1)$,因此总复杂度为$O(n)$。

\section{双调序列最大值问题}
\subsection*{问题分析}

双调序列的最大值一定位于增减区间的转换点,且先增后减,因此,如果可以快速确定增减区间转折点出现的位置区间,则可以快速的解决问题。

采用倍增的方法从左向右搜索数列的由增到减的拐点,会出现以下两种情况

\begin{enumerate}
    \item 数列左端位于增区间内,由于双调函数的性质,若此增区间右侧还有另一个增区间,那么位于另一个增区间的数应均小于左端点所处的增区间的所有数。因此只需要搜索离左端增区间最近的拐点即可。
    搜索的步骤如下\begin{enumerate}
        \item 取倍增区间长度初始值为$d = 1$。
        \item 比较数列中 $A[l + d]$、$A[l + d + 1]$、$A[l + d/2]$ 若满足$A[l + d] < A[l + d + 1] \text{且} A[l + d] > A[l + d/2]$,则说明$A[l..l+d]$仍然位于增区间,还没找到拐点。
        \item 将d倍增,重复步骤2,直到条件不满足,说明拐点位于$A[l..l+d]$中。
        \item 递归的搜索$A[l..l+d]$。
    \end{enumerate}
    \item 数列左端位于减区间内,则按照情况1的方法,找到下一个增区间,从而将问题转化回情况1,搜索步骤同1中步骤,1.2的判断条件改为$A[l + d] > A[l + d + 1] \text{且} A[l + d] < A[l + d/2]$
\end{enumerate}

\subsection*{算法描述}

如算法\ref{algo:findPeak}。\\

\begin{algorithm}[H]
    \SetKw{Let}{Let}
    \SetKw{Var}{Var}
    \caption{$findPeak(l, r)$}\label{algo:findPeak}
    \KwIn{双调序列$A[l..r]$}
    \KwOut{$A$中最大值}
    \If(){$l=r-1$}{
        \Return{$A[l]$}\\
    }
    \If(){$A[l] < A[l+1]$}{
        $d \leftarrow 1$\\
        \While(){$l+d < r $ and $A[l + d] < A[l + d + 1] $ and $ A[l + d] > A[l + d/2]$}{
            $d \leftarrow 2d$
        }
        \Return{$findPeak(l+d/2, l+d)$}
    }\Else(){
        $d \leftarrow 1$\\
        \While(){$l+d < r $ and $A[l + d] > A[l + d + 1] $ and $ A[l + d] < A[l + d/2]$}{
            $d \leftarrow 2d$
        }
        \Return{$findPeak(l+d, r)$}
    }
\end{algorithm}

\subsection*{时间复杂度分析}

对于情况1
每次递归可以将问题规模减小一半,搜索区间采用倍增方法,因此有递推式
\begin{equation}
    \begin{aligned}
        T(1) &= 1\\
        T(n) & =T(n/2) + O(\log n)
    \end{aligned}
    \nonumber
\end{equation}
假设$n$为2的幂次
\begin{equation}
    \begin{aligned}
        T(n) &= T(n/2) + O(\log n)\\
            &= \log n + \log \frac{n}{2} + \log \frac{n}{4} + ...\\
            &= \log n * \log n - (\log 1 + \log 2 + \log 4 + ... + \log n)\\
            &= (\log n) ^ 2 - (0+1+2+3+...+\log n)\\
            &= (\log n)^2 - \frac{(1+\log n)\log n}{2}\\
            &= \log^2 n - \frac{1}{2}\log n - \frac{1}{2} \log^2 n\\
            &= \frac{1}{2} \log^2 n - \frac{1}{2} \log n\\
            &= O(log^2 n)
    \end{aligned}
    \nonumber
\end{equation}

对于情况2
需要$O(\log n)$的代价转化成情况1。

因此总体时间复杂度为$O(log^2 n) + O(\log n) = O(\log ^2 n)$。

\section{字符串等价关系判定问题}

\subsection*{问题分析}

字符串等价关系可以将字符串划分成等价类,只需要找出类中的代表元素,判断两个字符串的代表元素是否相等即可。

可以将一个等价类中字典序最小串的作为这个等价类的代表元素。问题转化为寻找与某个字符串等价的字典序最小的字符串。

\subsection*{算法描述}

如算法\ref{algo:findSymbol}

\begin{algorithm}[H]
    \SetKw{Let}{Let}
    \SetKw{Var}{Var}
    \caption{$findSymbol(A[1..n], B[1..m],k)$}\label{algo:findSymbol}
    \KwIn{$S[1..n]$}
    \KwOut{与$S$等价的字典序最小的串}
    \If(){$n\equiv 1 (mod 2)$}{
        \Return{$S[1..n]$}\\
    }\Else(){
        $S1 \leftarrow findSymbol(S[1..n/2])$\\
        $S2 \leftarrow findSymbol(S[n/2+1 .. n])$\\
        \If(){$S1 < S2$}{
            \Return{$S1+S2$}
        }\Else{
            \Return{$S2+S1$}
        }
    }
\end{algorithm}

\subsection*{时间复杂度分析}

每次递归可以将问题化简成规模为一半的子问题,字符串的比较需要$O(n)$的时间复杂度。

有
\begin{equation}
    \begin{aligned}
        T(1) &= 1\\
        T(n) &=2T(n/2)+O(n)
    \end{aligned}
    \nonumber
\end{equation}

由主定理可得时间复杂度为$O(n \log n)$。

比较两个代表串是否等需要$O(n)$。

因此最终时间复杂度为$O(n \log n) + O(n) = O(n \log n)$。


\section{向量的最小和问题}

\subsection*{问题分析}
由于每个向量可以取4种状态,因此,寻找两个向量最小和的问题可以转化成寻找两个向量最小差的问题。

考虑所有向量横纵坐标都为正的选择,若$v_{i}, v_{j}$两个向量距离最近,则$v_{i}, -v_{j}$两个向量和最小。

因此只需求出转化后的$n$个向量中距离最近的两个点即可。

将点按照x坐标排序,选择$x=m$作为分割直线,其中,$m$为$x$中各点坐标的中值,分割成$S1,S2$两个集合,可以递归地求解$S1$和$S2$中距离最小的两个点。
设$S1$和$S2$中最小距离为$d1$和$d2$, 设$d=min(d1,d2)$。

若S中还有距离小于$d$的点对,则必定分别属于$S1$和$S2$,不妨设$p$属于$S1$,$q$属于$S2$,那么$p$和$q$距离$x=m$的距离均小于$d$。

设$P1$为$x=m$左侧距离小于$d$的点集, $P2$为$x=m$右侧距离小于$d$的点集。则 $p \in P1, q \in P2$。

对于任意一个$P1$中的点,可能和这个点构成最小距离点的$P2$中的点只能位于一个 $d * 2d$的矩形中。由于矩形中任意两个点的距离都不小于$d$,
因此矩形内部最多存在6个点。只需要选择按$y$排序后离$p$最近的6个点比较即可。


\subsection*{算法描述}

如算法\ref{algo:findNearest}

\begin{algorithm}[H]
    \SetKw{Let}{Let}
    \SetKw{Var}{Var}
    \caption{$findNearest(S[1..n])$}\label{algo:findNearest}
    \KwIn{$S[m..n]$}
    \KwOut{S中距离最小的两个点$v_{i}, v_{j}$}
    $d \leftarrow \infty$\\

    \If(){$m+1=n$}{
        \Return{$S[m], S[n]$}
    }
    $mid \leftarrow (m+n)/2 $\\
    $t1, t2 \leftarrow findNearest(m,mid)$\\
    $d1 \leftarrow distance(t1,t2)$\\
    $t3, t4 \leftarrow findNearest(mid, n)$
    $d2 \leftarrow distance(t3, t4)$\\
    \If(){$d1 < d2$}{
        $v_i, v_j = t1, t2$\\
    }\Else(){
        $v_i, v_j = t3, t4$
    }
    $d \leftarrow min(d1, d2)$\\
    $k \leftarrow 1$ \\
    \For(){$i = m$ up to $n$}{ \emph{//find all point in d * 2d area} \\
        \If(){$S[i].x-S[mid].x < d$}{
            $temp[k] \leftarrow i$\\
            $k \leftarrow k+1$
        }
    }

    $sort(temp)$ \\
    \emph{//sorted temp by y form small to big} \\

    \For(){$i= 0$ up to $k$}{ \emph{//find the nearest two points}\\
        \For(){$j=j+1; j<k and S[temp[j]].y - S[temp[i]].y < d; j++$}{
            \If(){$d < distance(S[temp[i]], S[temp[j]])$}{
                $v_i, v_j \leftarrow S[temp[i]], S[temp[j]] $
            }
            $d=min(d, distance(S[temp[i]], S[temp[j]]))$
        }
    }

    \Return{$v_i, v_j$}
\end{algorithm}


\subsection*{时间复杂度分析}

每次分治可以将问题划分为两个规模为二分之一的子问题,若每次分治的同时使用归并排序,则合并的时间复杂度为$O(n)$。
递归式为

\begin{equation}
    \begin{aligned}
        T(2) &= 1\\
        T(n) & =2T(n/2) + O(n)
    \end{aligned}
    \nonumber
\end{equation}

由主定理可得最终时间复杂度$O(n \log n)$。