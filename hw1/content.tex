\section{请给出$T(n)$尽可能紧凑的渐进上界并予以说明,可假定$n$是2的整数次幂}

\subsection{}
\begin{equation}
    \begin{aligned}
        T(1) &=T(2)=1\\
        T(n) &=T(n-2)+1\quad if\quad n>2
    \end{aligned}
    \nonumber
\end{equation}

解:
\begin{equation}
    \begin{aligned}
        T(n) &=T(n-2)+1\\
             &=1+T(n-4)\\
             &=1+1+T(n-6)\\
             &=...\\
             &=n/2\\
             & <n\\
    \end{aligned}
    \nonumber
\end{equation}

因此上界为 $O(n)$    

\subsection{}
\begin{equation}
    \begin{aligned}
        T(1) &=1\\
        T(n) &=T(n/2)+1\quad if\quad n>1
    \end{aligned}
    \nonumber
\end{equation}

解:
根据主定理
\begin{equation}
    \begin{aligned}
        a=1,b=2,d=0\\
        d<\log_{b}a
    \end{aligned}
    \nonumber
\end{equation}

因此上界为 $O(\log n)$

\subsection{}
\begin{equation}
    \begin{aligned}
        T(1) &=1\\
        T(n) &=T(n/2)+n\quad if\quad n>1
    \end{aligned}
    \nonumber
\end{equation}

解:
根据主定理
\begin{equation}
    \begin{aligned}
        a=1,b=2,d=1\\
         d > \log_{b}a \\
    \end{aligned}
    \nonumber
\end{equation}

因此上界为 $O(n)$

\subsection{}
\begin{equation}
    \begin{aligned}
        T(1) &=1\\
        T(n) &=2T(n/2)+1\quad if\quad n>1
    \end{aligned}
    \nonumber
\end{equation}

解:
根据主定理
\begin{equation}
    \begin{aligned}
        a=2,b=2,d=0\\
         a \neq 1\\
         d < \log_{b}a
    \end{aligned}
    \nonumber
\end{equation}

因此上界为$O(n^{\log_{b}a})=O(n)$

\subsection{}
\begin{equation}
    \begin{aligned}
        T(1) &=1\\
        T(n) &=4T(n/2)+1\quad if\quad n>1
    \end{aligned}
    \nonumber
\end{equation}

解:
根据主定理
\begin{equation}
    \begin{aligned}
        a=4,b=2,d=0\\
         a \neq 1\\
         d < \log_{b}a
    \end{aligned}
    \nonumber
\end{equation}

因此上界为 $O(n^{\log_{2}4})=O(n^{2})$

\subsection{}
\begin{equation}
    \begin{aligned}
        T(1) &=1\\
        T(n) &=T(n/2)+ \log n\quad if\quad n>1
    \end{aligned}
    \nonumber
\end{equation}

解:
根据主定理
\begin{equation}
    \begin{aligned}
        a=1,b=2,f(n)=\log n =\Omega(n^{\log_{2}1})=\Omega(n) \\
    \end{aligned}
    \nonumber
\end{equation}

因此上界为 $O(\log n)$

\subsection{}
\begin{equation}
    \begin{aligned}
        T(1) &=1\\
        T(n) &=3T(n/2)+ n^{2} \quad if\quad n>1
    \end{aligned}
    \nonumber
\end{equation}

解:
根据主定理
\begin{equation}
    \begin{aligned}
        a=3,b=2, d=2 > \log_{2}3 \\
    \end{aligned}
    \nonumber
\end{equation}

因此上界为 $O(n^{2})$

\section{寻找中位数问题}

\subsection*{问题分析}

寻找中位数,也就是寻找AB两数组元素中第$(n+m+1)/2$小的数(当$m+n$为奇数)或第$(m+n/2)$ 和 第$(m+n+1)/2$个数(当$m+n$为偶数)。
不妨设目标数为$k$。有
\begin{equation}
    k = \left\{
    \begin{aligned}
        \frac{m+n+1}{2} \quad if \quad (m+n)\equiv 1 (mod 2) \\
        \frac{m+n}{2} and \frac{m+n+1}{2} \quad if \quad (m+n)\equiv 0 (mod 2) \\
    \end{aligned}
    \right.
    \nonumber
\end{equation}

比较$A[\lfloor k/2 \rfloor]$ 与 $B[\lfloor k/2 \rfloor]$ 

若 $A[\lfloor k/2 \rfloor] \leq B[\lfloor k/2 \rfloor]$ 说明$A[\lfloor k/2 \rfloor]$至多是第$k-1$ 小的元素, 因此$A[\lfloor k/2 \rfloor]$之前的元素不可能是中位数。

同理 若  $A[\lfloor k/2 \rfloor] > B[\lfloor k/2 \rfloor]$ 说明$B[\lfloor k/2 \rfloor]$至多是第$k-1$ 小的元素, 因此$B[\lfloor k/2 \rfloor]$之前的元素不可能是中位数。

下一步只需分析$A[\lfloor k/2 \rfloor + 1...n]$和$B[0...m]$中的第$\lceil k/2 \rceil$小的元素即可。 

\subsection*{算法描述}
如算法\ref{algo:mid}。

\begin{algorithm}
    \SetKw{Let}{Let}
    \SetKw{Var}{Var}
    \caption{$findKthMinElement(A[1..n], B[1..m],k)$}\label{algo:mid}
    \KwIn{$A[1..n], B[1..m],k$}
    \KwOut{第$k$小的元素}
     $p \leftarrow 1$\\
     $q \leftarrow 1$\\
    \While(){true}{
        \If(){$p = n$}{
            \Return{$B[k-1]$} 
        }
        \If(){$q = m$}{
            \Return{$A[k-1]$}
        }
        \If(){$k=1$}{
            \Return{$min(A[p],B[q])$}
        }

        $pNext\leftarrow min(p+k/2-1,m)$\\
        $qNext\leftarrow min(q+k/2-1,n)$\\

        \If(){$A[pNext] \leq B[qNext]$}{
            $k \leftarrow k-(pNext - p + 1)$\\
            $p\leftarrow pNext + 1$\\
        }\Else(){
            $k \leftarrow k-(qNext - q + 1)$\\
            $p \leftarrow qNext + 1$\\
        }
    }
    
\end{algorithm}

\subsection*{时间复杂度分析}
每次循环至少能将$k$减少一半,因此时间复杂度为$O(\log (m+n))$。

\section{双调序列最大值问题}
\subsection*{问题分析}

\subsection*{算法描述}

\subsection*{时间复杂度分析}

\section{字符串等价关系判定问题}

\subsection*{问题分析}

字符串等价关系可以将字符串划分成等价类,只需要找出类中的代表元素,判断两个字符串的代表元素是否相等即可。

可以将一个等价类中字典序最小串的作为这个等价类的代表元素。问题转化为寻找与某个字符串等价的字典序最小的字符串。

\subsection*{算法描述}

如算法\ref{algo:findSymbol}

\begin{algorithm}
    \SetKw{Let}{Let}
    \SetKw{Var}{Var}
    \caption{$findSymbol(A[1..n], B[1..m],k)$}\label{algo:findSymbol}
    \KwIn{$S[1..n]$}
    \KwOut{与$S$等价的字典序最小的串}
    \If(){$n\equiv 1 (mod 2)$}{
        \Return{$S[1..n]$}\\
    }\Else(){
        $S1 \leftarrow findSymbol(S[1..n/2])$\\
        $S2 \leftarrow findSymbol(S[n/2+1 .. n])$\\
        \If(){$S1 < S2$}{
            \Return{$S1+S2$}
        }\Else{
            \Return{$S2+S1$}
        }
    }
\end{algorithm}

\subsection*{时间复杂度分析}

每次递归可以将问题化简成规模为一半的子问题,字符串的比较需要$O(n)$的时间复杂度。

有
\begin{equation}
    \begin{aligned}
        T(n) &=2T(n/2)+O(n)
    \end{aligned}
    \nonumber
\end{equation}

得时间复杂度为$O(n \log n)$。

比较两个代表串是否等需要$O(n)$。

因此最终时间复杂度为$O(n \log n) + O(n) = O(n \log n)$。


\section{向量的最小和问题}

\subsection*{问题分析}
由于每个向量可以取4种状态,因此,寻找两个向量最小和的问题可以转化成寻找两个向量最小差的问题。

考虑所有向量横纵坐标都为正的选择,若$v_{i}, v_{j}$两个向量距离最近,则$v_{i}, -v_{j}$两个向量和最小。

因此只需求出转化后的$n$个向量中距离最近的两个点即可。

将点按照x坐标排序,选择$x=m$作为分割直线,其中,$m$为$x$中各点坐标的中值,分割成$S1,S2$两个集合,可以递归地求解$S1$和$S2$中距离最小的两个点。
设$S1$和$S2$中最小距离为$d1$和$d2$, 设$d=min(d1,d2)$。

若S中还有距离小于$d$的点对,则必定分别属于$S1$和$S2$,不妨设$p$属于$S1$,$q$属于$S2$,那么$p$和$q$距离$x=m$的距离均小于$d$。

设$P1$为$x=m$左侧距离小于$d$的点集, $P2$为$x=m$右侧距离小于$d$的点集。则 $p \in P1, q \in P2$。

对于任意一个$P1$中的点,可能和这个点构成最小距离点的$P2$中的点只能位于一个 $d * 2d$的矩形中。由于矩形中任意两个点的距离都不小于$d$,
因此矩形内部最多存在6个点。只需要选择按$y$排序后离$p$最近的6个点比较即可。


\subsection*{算法描述}

如算法\ref{algo:findNearest}

\begin{algorithm}
    \SetKw{Let}{Let}
    \SetKw{Var}{Var}
    \caption{$findNearest(S[1..n])$}\label{algo:findNearest}
    \KwIn{$S[m..n]$}
    \KwOut{S中距离最小的两个点$v_{i}, -v_{j}$的距离}
    $d \leftarrow \infty$
    \If(){$n=m$}{
        \Return{$d$}
    }
    \If(){$m+1=n$}{
        \Return{$distance(S[m], S[n])$}
    }
    $mid \leftarrow (m+n)/2 $\\
    $d1 \leftarrow findNearest(m,mid)$\\
    $d2 \leftarrow findNearest(mid, n)$\\
    $d \leftarrow min(d1, d2)$\\
    $k \leftarrow 1$ 
    \For(){$i = m up to n$}{
        \If(){$S[i].x-S[mid].x < d$}{
            $temp[k] \leftarrow i$\\
            $k \leftarrow k+1$
        }
    }
    $sort(temp)$\\
    \For(){$i= 0 up to k$}{
        \For(){$j=j+1; j<k and S[temp[j]].y - S[temp[i]].y < d; j++$}{
            $d=min(d, distance(S[temp[i]], S[temp[j]]))$
        }
    }
    \Return{$d$}
    

\end{algorithm}


\subsection*{时间复杂度分析}

若每次分治的同时使用归并排序,则合并的时间复杂度为$O(n)$。
递归式为$T(n)=2T(n/2) + O(n)$。
最终时间复杂度$O(n \log n)$。